\documentclass[12pt]{article}
\usepackage[a4paper,margin=2cm]{geometry}
\usepackage[utf8]{inputenc}
\usepackage[russian]{babel}
\usepackage{amsmath,amssymb,amsfonts}
\usepackage{enumitem}
\usepackage{titlesec}
\usepackage{fancyhdr}
\usepackage{tikz}
\usepackage{mathrsfs}
\usepackage{multicol}
\usepackage{amssymb}

\titleformat{\section}{\normalfont\Large\bfseries}{\thesection.}{0.5em}{}
\titleformat{\subsection}{\normalfont\large\bfseries}{\thesubsection}{0.5em}{}
\setlist[itemize]{noitemsep, topsep=0pt}

\begin{document}

\begin{center}
    {\LARGE \textbf{Линейная алгебра. Модуль: Евклидовы пространства. Тензоры}}\\
    \vspace{0.3em}
    Башков Иван\\
    \vspace{0.3em}
    2 июня 2025 г.
\end{center}

\section{Билинейные функции и их матрицы. Преобразование матрицы билинейной функции при замене базиса. Ранг и ядра билинейной функции. Связь между билинейной функцией и линейным отображением в сопряжённое пространство.}

\subsection{Билинейные функции и их матрицы}

\textbf{Определение.} \textit{Билинейной функцией (формой)} на векторном пространстве $V$ называется отображение 
\[
\beta: V \times V \to \mathbb{F},
\]
которое линейно по каждому аргументу.

Это означает, что:
\begin{itemize}
    \item $\beta(x_1 + x_2, y) = \beta(x_1, y) + \beta(x_2, y)$,
    \item $\beta(\lambda x, y) = \lambda \beta(x, y)$,
    \item $\beta(x, y_1 + y_2) = \beta(x, y_1) + \beta(x, y_2)$,
    \item $\beta(x, \lambda y) = \lambda \beta(x, y)$,
\end{itemize}
для всех $x, x_1, x_2, y, y_1, y_2 \in V$, $\lambda \in \mathbb{F}$.

\vspace{0.5em}
\textbf{Матрица билинейной формы.} Пусть $e_1, \dots, e_n$ — базис в $V$, тогда билинейная форма $\beta$ задаётся значениями $\beta(e_i, e_j)$, которые образуют матрицу $B = (\beta(e_i, e_j))$.

Если $x = \sum x_i e_i$, $y = \sum y_j e_j$, то:
\[
\beta(x, y) = X^\top B Y, \quad \text{где } X = (x_1, \dots, x_n)^\top,\ Y = (y_1, \dots, y_n).
\]

\subsection{Преобразование матрицы билинейной формы при замене базиса}

Пусть $C$ — матрица перехода от базиса $e$ к новому базису $\tilde{e}$. Тогда матрица билинейной формы в новом базисе $\tilde{B}$ связана с исходной по формуле:
\[
\tilde{B} = C^\top B C.
\]

\textbf{Доказательство:}
Пусть $C=(e \rightsquigarrow \tilde{e})$, где $e$ – старый базис, $\tilde{e}$ – новый базис. Тогда\\
$X = (e \rightsquigarrow \tilde{e}) \tilde{X} = C\tilde{X}$, $Y = (e \rightsquigarrow \tilde{e}) \tilde{Y} = C\tilde{Y}$\\
$\beta(x,y)=(C \tilde{X})^\top B(C\tilde{Y})=\tilde{X}^\top(C^\top BC)\tilde{Y}=\tilde{X}^\top B\tilde{Y} \implies \tilde{B} = C^\top BC, \ C \in GL_n(F) \ \square$

\textbf{Следствие.} Ранг билинейной формы не зависит от выбора базиса, так как $C$ невырожденная.

\subsection{Ранг и ядра билинейной функции}

\textbf{Определение.} \textit{Рангом билинейной формы} $\beta$ называется ранг её матрицы:
\[
\mathrm{rk}\, \beta = \mathrm{rk}\, B.
\]

\textbf{Правое ядро:}
\[
R_\beta =\mathrm{Ker} \beta = \{ y \in V \mid \beta(x, y) = 0 \ \ \forall x \in V \}.
\]

\textbf{Левое ядро:}
\[
L_ \beta = \{ x \in V \mid \beta(x, y) = 0 \ \ \forall y \in V \}.
\]

\textbf{Определение.} Форма называется \textit{невырожденной}, если $\mathrm{Ker} \beta = \{0\}$.

\textbf{Лемма.} $\beta{} \text{ невырождена } \Longleftrightarrow rk\beta=n.$

\subsection{Связь с линейным отображением в сопряжённое пространство}

\textbf{Теорема.} Билинейные формы на $V$ канонически изоморфны линейным отображениям из $V$ в его сопряжённое пространство $V^*$.

\textbf{Доказательство:}

1. Для каждой билинейной формы $\beta \in BL(V)$ определим отображение $\varphi(\beta): V \to V^*$ следующим образом:
     $$
     \varphi(\beta)(v) = \beta(v, -),
     $$
     где $\beta(v, -)$ — линейный функционал на $V$, определенный как:
     $$
     \beta(v, -)(w) = \beta(v, w), \quad \forall w \in V.
     $$

2. Линейность $\varphi$. Для любых $\beta_1, \beta_2 \in BL(V)$ и $\alpha \in F$:
       $$
       \varphi(\beta_1 + \alpha \beta_2)(v) = (\beta_1 + \alpha \beta_2)(v, -) = \beta_1(v, -) + \alpha \beta_2(v, -).
       $$
       Это означает:
       $$
       \varphi(\beta_1 + \alpha \beta_2) = \varphi(\beta_1) + \alpha \varphi(\beta_2).
       $$
       Таким образом, $\varphi$ — линейное отображение.

3. Ядро $\varphi$ состоит из всех билинейных форм $\beta$, для которых $\varphi(\beta) = 0$, то есть:
     $$
     \mathrm{Ker} \varphi = \{\beta \in BL(V) \mid \varphi(\beta) = 0\}.
     $$
     Если $\varphi(\beta) = 0$, то для любого $v \in V$:
     $$
     \beta(v, -) = 0,
     $$
     что означает, что $\beta(v, w) = 0$ для всех $v, w \in V$. Следовательно, $\beta = 0$. Таким образом:
     $$
     \mathrm{Ker} \varphi = \{0\}.
     $$

4. Образ $\varphi$ состоит из всех линейных отображений $V \to V^*$, которые можно получить из билинейных форм. Для любого отображения существует $\beta$, что $\varphi(\beta)$ – это именно такое линейное отображение. Таким образом:
     $$
     \mathrm{Im} \varphi \leq \mathrm{Hom}(V, V^*), \ \dim \mathrm{Im}\varphi=\dim \mathrm{Hom}(V,V^*)=\dim BL(V) \implies \mathrm{Im}\varphi = \mathrm{Hom}(V,V^*)
     $$

5. Поскольку $\mathrm{Ker} \varphi = \{0\}$ и $\mathrm{Im} \, \varphi = \mathrm{Hom}(V, V^*)$, отображение $\varphi$ является \textbf{изоморфизмом} между $BL(V)$ и $\mathrm{Hom}(V, V^*)$. $\ \ \square$



\section{Симметрические и кососимметрические билинейные функции, их матрицы. Ортогональное дополнение к подпространству относительно билинейной функции, его свойства.}

\subsection{Симметрические и кососимметрические билинейные функции, их матрицы}

\textbf{Определение.} Билинейная форма $\beta$ называется:
\begin{itemize}
    \item \textit{симметрической} $(BL^+(V))$, если $\beta(x, y) = \beta(y, x) \ \forall x, y \in V$;
    \item \textit{кососимметрической} $(BL^-(V))$, если $\beta(x, y) = -\beta(y, x) \ \forall x, y \in V$;
    \item \textit{антисимметрической (симплектической)}, если $\beta(x,x)=0 \ \forall x \in V$.
\end{itemize}

\textbf{Лемма.} Если $\beta$ антисимметрическая, то $\beta \in BL^-(V)$.

\textbf{Доказательство.}
$0 = \beta(x + y, x + y)
  = \beta(x, x) + \beta(x, y) + \beta(y, x) + \beta(y, y) = \beta(x, y) + \beta(y, x)
  \implies \beta(x, y) = -\beta(y, x).$ \\
В доказательстве использовалось $\beta(x,x)=0$. Это верно при $char \mathbb{F} \ne 2$: $\beta(x,x)=-\beta(x,x) \implies 2\beta(x,x)=0$.


\textbf{Лемма.} $BL(V) = BL^+(V) \oplus BL^-(V)$.

\textbf{Доказательство:}

\textbf{I.} $BL^+(V) \cap BL^-(V) = \{0\}$

Пусть $\beta \in BL^+(V) \cap BL^-(V)$. Тогда:
\[
\beta(x, y) = \beta(y, x) = -\beta(x, y) \implies 2\beta(x, y) = 0 \implies \beta = 0.
\]

\textbf{II.} Для любой билинейной формы $\beta(x, y)$ можно записать:
\[
\beta(x, y) = \frac{\beta(x, y) + \beta(y, x)}{2} + \frac{\beta(x, y) - \beta(y, x)}{2}.
\]
Здесь:
\[
\frac{\beta(x, y) + \beta(y, x)}{2} \in BL^+(V), \quad \frac{\beta(x, y) - \beta(y, x)}{2} \in BL^-(V). \ \ \square
\]

\textbf{Матрица билинейной формы.}
Пусть $B = (\beta(e_i, e_j))$ — матрица формы в базисе $e_1, \dots, e_n$.
\begin{itemize}
    \item $\beta$ симметрическая $\iff B = B^\top$;
    \item $\beta$ кососимметрическая $\iff B = -B^\top$.
\end{itemize}

\textbf{Следствие.} Ранг кососимметрической формы — чётное число.

\subsection{Ортогональное дополнение относительно билинейной формы}

Пусть $\beta$ – симметрическая/кососимметрическая билинейная форма. Векторы $x$ и $y$ \textbf{ортогональны} относительно $\beta$, если $\beta(x,y)=0$.

\textbf{Определение.} Пусть $U \leq V$. Ортогональным дополнением $U^\perp$ относительно формы $\beta$ называется:
\[
U^\perp = \{ y \in V \mid \beta(x, y) = 0 \ \forall x \in U \}.
\]

\textbf{Лемма.} Если $\beta$ невырожденная, то:
\begin{itemize}
    \textbf{I.} $\dim U^\perp = \dim V - \dim U$;\\
    \textbf{II.} $(U^\perp)^\perp = U$.
\end{itemize}

\textbf{Доказательство:}

\textbf{I.} Пусть $ e_1, \dots, e_k, e_{k+1}, \dots, e_n $ — базис пространства $ V $ и $ e_1, \dots, e_k $ — базис подпространства $ U $.

Рассмотрим ортогональное дополнение $ U^\perp $:
\[
U^\perp = \{ y \in V \mid \beta(e_i, y) = 0 \quad \forall i = 1, \dots, k \}.
\]

Пусть $ y = e_1 y_1 + \dots + e_n y_n $. Тогда система уравнений для $ y \in U^\perp $ имеет вид:
\[
\begin{cases}
b_{11} y_1 + \dots + b_{1n} y_n = 0 \\
b_{21} y_1 + \dots + b_{2n} y_n = 0 \\
\vdots \\
b_{k1} y_1 + \dots + b_{kn} y_n = 0
\end{cases}
\]

Это система линейных алгебраических уравнений (СЛАУ). Ранг матрицы системы равен рангу матрицы коэффициентов:
\[
\mathrm{rk} \begin{pmatrix}
b_{11} & \dots & b_{1n} \\
b_{21} & \dots & b_{2n} \\
\vdots & \ddots & \vdots \\
b_{k1} & \dots & b_{kn}
\end{pmatrix}.
\]

Ранг матрицы коэффициентов, в свою очередь, равен $\dim U$, так как $\beta$ невырождена $\implies$ любая подсистема строк в $B$ линейно независима.

Таким образом, размерность $ U^\perp $ вычисляется как:
\[
\dim U^\perp = n - \mathrm{rk} \begin{pmatrix}
b_{11} & \dots & b_{1n} \\
b_{21} & \dots & b_{2n} \\
\vdots & \ddots & \vdots \\
b_{k1} & \dots & b_{kn}
\end{pmatrix} = n - k = \dim V - \dim U.
\]

\textbf{II.} 1. Проверим включение $ U \subseteq (U^\perp)^\perp $:
   - Возьмём произвольный вектор $ u \in U $. По определению $ U^\perp $, для любого $ y \in U^\perp $ выполнено $ \beta(u, y) = 0 $.
   Следовательно, $ u \in (U^\perp)^\perp $, так как $ u $ ортогонален всем векторам из $ U^\perp $.

2. Вычислим размерность $ (U^\perp)^\perp $:
   по первой части, $ \dim U^\perp = n - k $.
   Аналогично, $ \dim (U^\perp)^\perp = \dim V - \dim U^\perp = n - (n - k) = k $.
   Так как $ \dim U = k $, получаем $ \dim (U^\perp)^\perp = \dim U $.

3. Поскольку $ U \subseteq (U^\perp)^\perp $ и размерности совпадают, имеем $ U = (U^\perp)^\perp. \ \ \square$


\textbf{Определение.} Подпространство $U$ называется \textit{невырожденным}, если ограничение формы $\beta|_U$ — невырожденное.

\textbf{Лемма.} $U$ невырожденное $\iff V = U \oplus U^\perp$.

\textbf{Доказательство.}

$\dim U = k, \ \dim U^\perp \geq n-k$

$\Longrightarrow$

$\dim (U+U^\perp)=\dim U + \dim U^\perp - \dim (U \cap U^\perp)$

$U \cap U^\perp = \set {0}, \dim(U+U^\perp)=\dim U+\dim U^\perp \geq k + (n-k) \geq n = \dim V$

$U+U^\perp \leq V \implies V = U \oplus U^\perp$

\Longleftarrow

$V = U \oplus U^\perp \implies V = U + U^\perp$

$n = \dim (U+U^\perp) = \dim U + \dim U^\perp \implies \dim(U \cap U^\perp)=0 \implies U \cap U^\perp = \set{0} \ \ \square$

\textbf{Определение.} Базис $e_1, \dots, e_n$ называется \textit{ортогональным}, если $\beta(e_i, e_j) = 0$ при $i \ne j$.

\textbf{Теорема.} Для любой симметрической билинейной формы на конечномерном пространстве существует ортогональный базис.

\textbf{Доказательство.} Индукция по $n = \dim V$:
\begin{itemize}
    \item База: $n = 1$ — очев.
    \item Шаг: $n = k$. Пусть $e_k \in V$ так, что $\beta(e_k, e_k) \ne 0$. Тогда $\beta$ невырождена на $U = \langle e_k \rangle$, и по невырожденности $\beta|_U$ имеем разложение $V = U \oplus U^\perp$, при этом $\dim U^\perp=k-1$. По предположению индукции в $U^\perp$ существует ортогональный базис. Добавим к нему $e_k$ и получим базис $V$.$\ \ \square$
\end{itemize}

\section{Квадратичные функции, поляризация. Канонический и нормальный виды симметрической билинейной и квадратичной функций.}

\subsection{Квадратичные функции, поляризация}

\textbf{Определение.} Пусть $\beta \in BL^+(V).$ \textit{Квадратичная форма} – $q(x)=\beta(x,x)$:

$$
q(x) =\sum_{1 \leq i \leq j \leq n}{x_i b_{ij}x_j}
$$

\textbf{Поляризация квадратичной формы} — восстановление билинейной формы:

$$
q(x + y) = \beta(x + y, x + y)
         = \beta(x, x) + 2\beta(x, y) + \beta(y, y)
$$

$$
\beta(x, y) = \frac{q(x + y) - q(x) - q(y)}{2}
$$

\subsection{Канонический и нормальный виды}

Для любой симметрической билинейной формы существует базис, в котором она имеет вид:
$$
\beta(x, y) = \sum_{i=1}^n a_i x_i y_i.
$$

Соответствующая квадратичная форма:
$$
q(x) = \sum_{i=1}^n a_i x_i^2.
$$

Нормальный вид — канонический вид, в котором все ненулевые коэффициенты равны $\pm 1$.
\begin{center}
    {\LARGE \textbf{Линейная алгебра. Модуль: Евклидовы пространства. Тензоры}}\\
    \vspace{0.3em}
    Малиновский Степан\\
    \vspace{0.3em}
    4 июня 2025 г.
\end{center}

\section{Расстояние и угол между вектором и подпространством}

\subsection*{Расстояние от вектора до подпространства}
Расстоянием от вектора $x$ до подпространства $L$ называется:
\[
d(x, L) = \min_{y \in L} \|x - y\|
\]
Минимум достигается при $y = \operatorname{proj}_L(x)$, поэтому:
\[
d(x, L) = \|x - \operatorname{proj}_L(x)\| = \|\operatorname{ort}_L(x)\|
\]
Таким образом, расстояние — это длина ортогональной составляющей.

\subsection*{Метрика в пространстве}
Пусть $V$ — векторное пространство, а $p: V \times V \to \mathbb{R}$ — функция, определяющая расстояние между векторами:
\[
p(x, y) = \|x - y\|
\]
Свойства метрики $p$:
\begin{enumerate}
    \item $p(x, y) = p(y, x)$
    \item $p(x, y) \geq 0$ и $p(x, x) = 0$
    \item $p(x, y) \leq p(x, t) + p(t, y)$
\end{enumerate}
Расстояние между множествами $A$ и $B$:
\[
p(A, B) = \inf \{ p(x, y) \mid x \in A, y \in B \}
\]

\subsection*{Теорема}

Пусть \( U \) — подпространство векторного пространства \( V \), а \( x \in V \). Тогда:

$\rho (x, U) = |ort_U(x)|$ и $pr_U(x)$ - единственный ближайший к x элемент U 

Докажем это утверждение:
Пусть $y \in U$:
\[
\rho (x - y)^2 = (x - y, x-y) = |x-y|^2 = |\operatorname{ort}_U(x) + \operatorname{pr}_U(x) - y|^2 = 
\]
\[
= |(pr_U(X) - y) + ort_U(x)|^2 = |pr_U(x) - y|^2 + |ort_U(x)|^2 \geq |ort_U(x)|^2; 
\]
и достигается при $y = pr_U(x)$

\subsection*{Угол между вектором и подпространством}
Углом $\varphi$ между вектором $x$ и подпространством $L$ называется угол между $x$ и его проекцией $\operatorname{proj}_L(x)$. Формула:
\[
\cos \varphi = \frac{\|\operatorname{proj}_L(x)\|}{\|x\|}
\]
\[
\sin \angle(x, L) = \frac{\|\operatorname{ort}_L(x)\|}{\|x\|}
\]
\[
\cos \angle(x, L) = \frac{(x, u)}{\|x\| \|u\|}, \quad u \in L, \; u \perp \operatorname{pr}_L(x)
\]

##### \textbf{Доказательство формулы для угла:}
Пусть $x$ — произвольный вектор, а $U$ — подпространство. Разложим $x$ по базису подпространства $U$ и его дополнения:
\[
x = \operatorname{pr}_U(x) + \operatorname{ort}_U(x),
\]
где $\operatorname{pr}_U(x) \in U$ — проекция $x$ на $U$, а $\operatorname{ort}_U(x) \in U^\perp$ — ортогональная составляющая.

Выберем произвольный вектор $u \in U$ и разложим его:
\[
u = \lambda \operatorname{pr}_U(x) + \mu u^\perp,
\]
где $u^\perp \in U^\perp$ — ортогональный компонент.

Тогда скалярное произведение $(x, u)$ вычисляется следующим образом:
\[
(x, u) = (\operatorname{pr}_U(x) + \operatorname{ort}_U(x), \lambda \operatorname{pr}_U(x) + \mu u^\perp) = \lambda \|\operatorname{pr}_U(x)\|^2.
\]

Норма вектора $u$ равна:
\[
\|u\| = \sqrt{\|\lambda \operatorname{pr}_U(x)\|^2 + \|\mu u^\perp\|^2}.
\]

Теперь вычислим косинус угла:
\[
\cos \angle(x, U) = \frac{(x, u)}{\|x\| \|u\|} = \frac{\lambda \|\operatorname{pr}_U(x)\|^2}{\|x\| \sqrt{(\lambda \operatorname{pr}_U(x))^2 + (\mu u^\perp)^2}}.
\]

Но $(\mu u^\perp)^2$ = 0, тогда получим:

\[
\cos \angle(x, U) = \frac{\lambda \|\operatorname{pr}_U(x)\|^2}{\|x\| \|\lambda \operatorname{pr}_U(x)\|} = \frac{\|\operatorname{pr}_U(x)\|}{\|x\|}.
\]

\subsection*{Теорема косинусов}
Для любых векторов $x$ и $y$:
\[
(x-y, x+y) = \|x\|^2 + \|y\|^2 - 2 \|x\| \|y\| \cos \angle(x, y)
\]
Следствие (теорема Пифагора):
\[
x \perp y \implies \|x-y\|^2 = \|x\|^2 + \|y\|^2
\]

\subsection*{Метрика в пространстве}
Пусть $V$ — векторное пространство, а $p: V \times V \to \mathbb{R}$ — функция, определяющая расстояние между векторами:
\[
p(x, y) = \|x - y\|
\]
Свойства метрики $p$:
\begin{enumerate}
    \item $p(x, y) = p(y, x)$
    \item $p(x, y) \geq 0$ и $p(x, x) = 0$
    \item $p(x, y) \leq p(x, t) + p(t, y)$
\end{enumerate}
Расстояние между множествами $A$ и $B$:
\[
p(A, B) = \inf \{ p(x, y) \mid x \in A, y \in B \}
\]

\section{Комплексификация евклидова пространства и эрмитово пространство}

\subsection*{Комплексификация евклидова пространства}
Пусть $V$ — вещественное евклидово пространство с скалярным произведением $(\cdot, \cdot): V \times V \to \mathbb{R}$.  
Комплексификацией $V$ называется комплексное линейное пространство:
\[
V_\mathbb{C} = V \otimes \mathbb{C}
\]
Элементы $V_\mathbb{C}$ имеют вид $x + iy$, где $x, y \in V$.

На $V_\mathbb{C}$ можно ввести эрмитово скалярное произведение:
\[
(a + bi, c + di)_\mathbb{C} = (a, c) + (b, d) + i((b, c) - (a, d)),
\]

Оно положительно определено:

\[
(z, z)_\mathbb{C} = (a + вi, a + bi)_\mathbb{C} = (a, a) + (b, b) \geq 0
\] и = 0 <=> z = 0

\subsection*{Свойства эрмитова скалярного произведения}

Докажем основные свойства эрмитова скалярного произведения:

\subsubsection{Эрмитова симметричность:}
\[
(x, y)_\mathbb{C} = \overline{(y, x)_\mathbb{C}}
\]

Пусть $x = a + bi$ и $y = c + di$, где $a, b, c, d \in V$. Тогда:
\[
(x, y)_\mathbb{C} = (a + bi, c + di)_\mathbb{C} = (a, c) + (b, d) + i((b, c) - (a, d)).
\]

Аналогично:
\[
(y, x)_\mathbb{C} = (c + di, a + bi)_\mathbb{C} = (c, a) + (d, b) + i((d, a) - (c, b)).
\]
Отсюда следует доказываемое

\subsubsection{Аддитивность по первому аргументу (по второму аналогично):}
\[
((a_1 + b_1i) + (a_2 + b_2i), c + di) = ((a_1 + a_1) + (b_1 + b_2)i, c + di) = (a_1 + a_2, c) + (b_1 + b_2, d) + i((b_1 + b_2, c) - (a_1+a_2, d)) = \]
\[ 
= (a_1, c) + (a_2, c) + (b_1,d) + (b_2,d) + i((b_1,c)+(b_2,c) - (a_1,d) - (a_2, d)) = (a_1+b_1i, c+di)+(a_2+b_2i,c+di)
\]

\subsubsection{Однородность по первому аргументу:}
\[
(a+bi, (q+wi)(c+di))_\mathbb{C}= (a+bi, (qc - wd) + i(qd + wc)) = (a, qc - wd) + (b, qd + wc) + i((b, qc - wd) - (a, qd + wc))=
\]
\[
= q(a,c)- w(a,d)+ q(b,d)+w(b,c) + i (q(b,c) - w(b,d) - q(a,d) - w (a,c))=
(q-wi)(a+bi, c+di)
\]
Но полуоднородность по второму аргументу: $(x, \alpha y)_\mathbb{C} = \overline{\alpha}(x,y)_\mathbb{C}$

$\beta \in BL(V_\mathbb{C})$ - полулинейная форма

\subsection*{Итог:}
$(V_\mathbb{C}, (  $ _$ , $_$ )_\mathbb{C})$ - Эрмитово (унитарное) пространство

\subsection{Неравенства Коши-Буняковского в  эрмитовом пространстве:}

Пусть H — эрмитово (унитарное) пространство с эрмитовым скалярным произведением $(*, *) : H\times H \to \mathbb{C}.$ Тогда для любых $x, y \in H$ выполняется следующее неравенство:
\[
|(x,y)| \leq \|x\| * \|y\|
\]

Доказательство:
\[\]
Рассмотрим функцию:
\[
f(t) = \|x + ty\|^2 = (x + ty, x + ty),
\]
где $t \in \mathbb{C}$. Эта функция всегда неотрицательна, так как норма квадратична:
\[
f(t) = (x, x) + t(y, x) + \overline{t}(x, y) + |t|^2(y, y).
\]

Пусть $t = -\frac{(x, y)}{(y, y)}$. Тогда:
\[
f\left(-\frac{(x, y)}{(y, y)}\right) = \|x\|^2 - \frac{|(x, y)|^2}{(y, y)} \geq 0.
\]

Отсюда:
\[
\|x\|^2 \cdot (y, y) \geq |(x, y)|^2,
\]
или:
\[
|(x, y)| \leq \|x\| \cdot \|y\|.
\]

\subsection*{Неравенство треугольника}

Для любого эрмитового пространства $H$ с эрмитовым скалярным произведением $(\cdot, \cdot): H \times H \to \mathbb{C}$ и нормой:
\[
\|x\| = \sqrt{(x, x)},
\]
выполняется неравенство треугольника:
\[
\|x + y\| \leq \|x\| + \|y\|.
\]

\subsubsection{Доказательство}
Рассмотрим квадрат нормы суммы:
\[
\|x + y\|^2 = (x + y, x + y) = (x, x) + (x, y) + (y, x) + (y, y) = \|x\|^2 + \|y\|^2 + (x, y) + \overline{(x, y)}.
\]

Заметим, что $(x, y) + \overline{(x, y)} = 2 \operatorname{Re}(x, y)$, значит:
\[
\|x + y\|^2 = \|x\|^2 + \|y\|^2 + 2 \operatorname{Re}(x, y).
\]

Так как $\operatorname{Re}(x, y) \leq |(x, y)|$, получаем:
\[
\|x + y\|^2 \leq \|x\|^2 + \|y\|^2 + 2 |(x, y)|.
\]

Применим неравенство Коши-Буняковского:
\[
|(x, y)| \leq \|x\| \cdot \|y\| \implies \|x + y\|^2 \leq \|x\|^2 + \|y\|^2 + 2 \|x\| \|y\| = (\|x\| + \|y\|)^2.
\]

Извлекая корень из обеих частей:
\[
\|x + y\| \leq \|x\| + \|y\|.
\]

\section{Соответствие между операторами и билинейными функциями в евклидовом пространстве. Со
пряжённый оператор и его свойства}

\subsection*{Построение изоморфизма между оператором и билинейной формой}

Пусть V - конечномерное векторное пространство, $dim V < \infty$, $\beta \in BL(V)$

Рассмотрим отображение: $V \to V^*$, $y \in V \to \beta( $_$ , y)$

$B \in Hom(V, V^*)$ (линейность по второму аргументу)

$y_1 + y2 \to \beta($_$, y_1) + \beta($_$, y_2) = \beta($_$, y_1 + y_2)$

Тогда можно определить отображение:
$\varphi: BL(V) \to Hom(V, V^*), \varphi$ линейно по определению операций в $BL(V) и Hom(V,V^*)$ 

$Ker(\varphi) = {0} - \varphi$ инъективно 
$dim(Im(\varphi)) = n^2 - 0 = n^2$
$Im(\varphi) = Hom(V,V^*) - \varphi$ сюръективно => $\varphi$ - канонический изоморфизм

\subsection*{Теорема}
$BL(V) = Hom(V, V^*)$
\[\]
Для любого $ x \in V, x \ne 0, X^TBY \ne 0$
\[\]
Невырожденный $BL(V) = Iso(V,V^*)$

\subsection*{Следствие}
V - евклидово $=> V = V ^*$
\[\]
$e_1,...,e_n -$ ортонормированный базис в V
\[\]
$e_1^*,...,e_n^* -$ сопряженный базис в $V^*$
\[\]
$(e_i, x) = (e_i, x_1e_1 + ...+x_ne_n) = x_i, (e_i,e_j) = \delta _{ij} = e_i^*(e_j)$ и $=e_j(e_i) => e_i^* = e_i$

\subsection*{Теорема о каноническом изоморфизме}
Пусть $V$ — конечномерное евклидово пространство. Тогда существует канонический изоморфизм:
\[
BL(V) \cong End(V)
\]

\subsection*{Доказательство}
$\alpha(x,y) = (x, Ay)$
\[\]
\textbf{Гомоморфизм:}
\[
(x, (A_1 + A_2)y) = (x, A_1 y) + (x,A_2y) = \alpha_1(x,y)+ \alpha_2(x,y) = (\alpha_1 + \alpha_2)(x,y)
\]
\[
A_1 + A_2 \longleftrightarrow \alpha_1 + \alpha_2 => dim(End(V)) = dim (BL(V)) = n^2
\]
\textbf{Ядро}:
\[\]
$A \ne 0$ существует $y \in V, A(y) \ne 0, \alpha(x,y) \ne 0 => \alpha \not\equiv 0$ - только нулевые 
\[\]
Значит, $End(V) \simeq BL(V)$ (канонический)

\subsection*{Следствие}
 В ортонормированном базисе евклидового пространства М матрица
 $A \in End(V)$ и соответствующая ему билинейная формы $\alpha \in BL(V)$ совпадют
 \[\]
 $\alpha^* (x,y) = \alpha(y,x)$ - сопряженная билинейная форма 
 \[\]
 $(x, Ay) = (A^*x,y)$ - определение сопряженного оператора в евклидовом пространстве

 \subsection*{Лемма}
 e - ортонормированный бизис => $[A^*]_e = [A]_e^T$

 \subsection*{Определение}
 Оператор A в евклидовом пространстве самосопряженный, если $A^* = A$ => $(x,Ay)= (Ax,y)$

\subsection*{Свойства}
\subsubsection*{
$1. (A+B)^* = A^*+B^*$}
\subsubsection*{
$2. E^* = E$}
\subsubsection*{
$3. (\lambda A)^* = \lambda A^*, \lambda \in \mathbb{R}$}
\subsubsection*{
$4. (AB)^* = B^*A^*$}

\[\]

\section{Ортогональные операторы, их свойства. Канонический вид матрицы ортогонального оператора.
 Собственные значения и собственные векторы ортогонального оператора. Комплексификация
 ортогонального оператора унитарный оператор. Собственные значения и собственные вектора
 унитарного оператора.}

\subsection*{Ортогональные операторы, их свойства}

Пусть $ V $ — евклидово пространство с вещественным скалярным произведением $ (\cdot, \cdot) $, и пусть $ A: V \to V $ — линейный оператор. Рассмотрим следующие утверждения:

\begin{enumerate}
    \item $ (Ax, Ay) = (x, y) $ для всех $ x, y \in V $
    \item $ \|Ax\| = \|x\| $ для всех $ x \in V $
    \item $ A \in \operatorname{Aut}(V) $, изоморфизм евклидовых пространств = биективный изоморфизм
    \item $ A^* = A^{-1} $
    \item $ A^T A = I $
\end{enumerate}

Докажем их эквивалентность.

\subsubsection*{1. $ 1 \Rightarrow 2 $}

Пусть $ (Ax, Ay) = (x, y) $ для всех $ x, y \in V $. Положим $ y = x $:
$$
(Ax, Ax) = (x, x)
\Rightarrow \|Ax\|^2 = \|x\|^2
\Rightarrow \|Ax\| = \|x\|
$$
Таким образом, оператор $ A $ сохраняет нормы.

\subsubsection*{2. $ 2 \Rightarrow 3 $}
$($_$,$_$) \in BL^+(V)$
\[\]
$(Ax,Ax) = |Ax|^2 \xrightarrow{поляризация} (Ax, Ay) = \frac{|Ax + Ay| - |Ax|^2 - |Ay|^2}{2} = \frac{|x+y|^2 - |x|^2 - |y|^2}{2} = (x,y)$

\subsubsection*{3. $ 1, 2\Rightarrow 3 $}

Найдем $x \in KerA$.\newline
$Ax = 0 : |Ax|^2 = |x|^2 = 0 = (x,x) =0 \Rightarrow x = 0$ \newline
Инъективность: KerA = {0} \newline
$dim(Im(A)) + dim(Ker(A)) = dim(V)$ \newline
Получает, что $Im(A) \leq V \Rightarrow V = Im(A) -$ сюръекция

\subsubsection*{4. $ 3 \Rightarrow 2 $}
Изометрия по определению изоморфизма

\subsubsection*{5. $ 1 \Longleftrightarrow 4 $}
$(Ax, Ay) = (x,y)$\newline
$(Ax, Ay) = (x,A^*Ay) = (x, Ey) = (x,y)$\newline
Получаем, что для любых $x,y \in V, ($_$,$_$)$ невырожденная =>$ A^*A = E \Longleftrightarrow A* = A^-1$

\subsubsection*{5. $ 4 \Longleftrightarrow 5 $}
Пусть e - ортонормированный базис, тогда:\newline
$[A^*]_e = A^T$, получим $A^TA = [A^*A] = [E] \Longleftrightarrow A \in \mathbb{O}_n(\mathbb{R})$

$$
\boxed{
\text{Ортогональный оператор — это оператор, удовлетворяющий условиям выше.}
}
$$
\subsection*{Собственные значения и собственные векторы ортогонального оператора}
\subsubsection*{Собственные значения ортогонального оператора}
Пусть x - собственный вектор $x \in V_\lambda$ \newline
$(Ax, Ax) = (x,x) = (\lambda x, \lambda x) = \lambda^2(x,x) => (\lambda^2 - 1)(x,x) = 0,$\newline но $(x,x) \ne 0 => \lambda^2 = 1 => \lambda = \pm 1$\newline
Если рассматривать комплексную плоскость, то $|\lambda| = 1$ означает, что решения лежат на единичной окружности

\subsubsection*{Собственные векторы ортогонального оператора}
\textbf{Лемма}\newline
Собственные векторы, соответствующие разным собственным значениям, ортогональны:\newline
Пусть $U(x) = \lambda x, U(y)= \mu *y $ и $\lambda \ne \mu$\newline
Рассмотрим скалярное произведение:\newline
$(U(x),U(y)) = (\lambda x, \mu x) = \lambda \mu (x,y)$\newline
$(U(x),U(y)) = (x,y)$ в силу ортогональности\newline
Получаем: $(\lambda \mu - 1)(x,y) = 0$\newline
Тогда возможен только один вариант, $(x,y) = 0 => $ ортогональны
\[\]
\textbf{Лемма}\newline
Вещественные собственные векторы: Существуют в вещественном пространстве тогда и только тогда, когда у оператора есть вещественные собственные значения (+1 или -1). Они соответствуют неподвижным точкам ($\lambda$ = 1) или точкам, отраженным относительно начала координат ($\lambda$ = -1).

\subsection*{Канонический вид матрицы ортогонального оператора}
\textbf{Теорема О каноническом виде}\newline
Для любого ортогонального оператора U, действующего в конечномерном вещественном евклидовом пространстве E размерности n, существует ортонормированный базис, в котором матрица оператора U имеет следующую блочно-диагональную каноническую форму:
$[U] = $\begin{pmatrix}
    R(\theta_1)&0&0&.&.&.&.&0\\
    0&R(\theta_2)&0&.&.&.&.&0\\
    0&...&...&.&.&.&.&0\\
    0&...&...&R(\theta_k&.&.&.&0\\
    0&...&...&... &\epsilon_1&.&.&0\\
    0&...&...& ...&...&\epsilon_2&.&0\\
    0&...&...& ...&...&.&.&0\\
    0&...&...& ...&...&.&.&\epsilon_m\\
\end{pmatrix}
\[\]
Где $\epsilon_i$ имеет размер 1x1 и равен $\pm1$, а блок $R(\theta_k)$ имеет размер 2x2 и вид:\newline

$R(\theta_k)$ = \begin{pmatrix}
    \cos(\theta_k) & -\sin(\theta_k)\\
    \sin(\theta_k) & \cos(\theta_k)
\end{pmatrix}
\[\]
Где $\theta_k \in (0, \pi) \cup (\pi, 2\pi)$\newline

Каждый блок $R(\theta_k)$ действует в своей двумерной инвариантной плоскости, в которой оператор U осуществляет поворот на угол $\theta_k$\newline

Количество блоков k может быть от 0 до n/2\newline

Каждому блоку $\epsilon_i$ = +1 соответствует одномерное инвариантное подпространство (собственный вектор $x_i$), на котором оператор U действует как тождественное преобразование: $U(x_i) = x_i$\newline

Каждому блоку $\epsilon_i$ = -1 соответствует одномерное инвариантное подпространство (собственный вектор xᵢ), на котором оператор U действует как центральная симметрия (отражение): $U(x_i) = -x_i$\newline

Количество таких блоков m может быть от 0 до n\newline
\textbf{Доказательство}\newline

Индукция по размерности:\newline
База: n = 1\newline
$U = <x>, x\ne 0$, тогда $|Ax| = |\lambda x|= |\lambda||x| = |x| => \lambda = \pm 1 $\newline
n = 2\newline
Пусть $e_1, e_2 -$ ортонормированный базис, $A \in \mathbb{O}(U):$\newline
$(A|_U)(e_1) = \alpha e_1 + \beta e_2$\newline
$|\alpha e_1 + \beta e_2|^2 = |e_1|^2 = 1 = \alpha^2 + \beta^2$\newline

$(A|_U)(e_2) = \gamma e_1 + \delta e_2$\newline
$|\gamma e_1 + \delta e_2|^2 = |e_1|^2 = 1 = \gamma^2 + \delta^2$\newline

$(A|_U)(e_1) \perp (A|_U)(e_2) => (\alpha e_1 + \beta e_2, \gamma e_1 + \delta e_2) = \alpha\gamma + \beta\delta = 0$\newline

\begin{cases}
    \alpha^2 + \beta^2 = 1 \\
    \gamma^2 + \delta^2 = 1
\end{cases}
\[\]

Пусть $\alpha, = \cos(\varphi), \beta = \sin(\varphi), \gamma = \cos(\psi), \delta = \sin(\psi)$, тогда:\newline

$\cos(\varphi)\cos(\psi) + \sin(\varphi)\sin(\psi) = 0$\newline
$\cos(\varphi - \psi) = 0$ => имеем вид:
\[\]
$[A|_U]_{e_1,e_2}$ = \begin{pmatrix}
    \cos(\varphi) & \sin(\varphi)\\
    \sin(\varphi) & \cos(\varphi)
\end{pmatrix}
\[\]
U A - инвариантно => $U^T$  $A^*$ - инвариантно => $U^\perp$ инвариантно относительно $A^{-1} \in \mathbb{O(U)}$ =>\newline
$A^{-1}: U^\perp \leadsto U^\perp$\newline
$A : U^\perp \leadsto U^\perp $\newline
$A^{-1} \in Aut(U)$\newline
Вывод $U^T A$ - инвариантно\newline
$V = U \bigoplus U^\perp, dim(U^\perp) <n$ по индукционному предположению в $U^\perp$ существует канонический вид\newline
\textbf{Единственность}\newline
$X_A = \prod^r_{i = 1} ((t^2 - 2t\cos(\varphi_i) + 1)(t-1)^q(t+1)^p$\newline
Все эти корни определенены однозначно с точностью до перестановки блоков





\subsection*{Комплексификация
 ортогонального оператора - унитарный оператор}
Пусть $V \rightsquigarrow (V_\mathbb{C}, ($_$,$_$)_\mathbb{C})$ - эрмитово пространство\newline

Тогда $A \in \mathbb{O}(V) \rightsquigarrow A_\mathbb{C} $ - сохраняет эрмитово скалярное произведение
\[
(A_\mathbb{C}(a+bi),A_\mathbb{C}(c+di))_\mathbb{C} = (A(a) + A(b)i, A(c) + A(d)i)_\mathbb{C} = 
\]
\[= (A(a), A(c)) + (A(b), A(d)) + i((A(b), A(c))-(A(a),A(d)) =
\]
\[
= (a,c) + (b,d) + i((b,c) - (a,d)) = (a+bi, c+di)_\mathbb{C}
\]
Получаем, что $A_\mathbb{C}$ - унитарный оператор, $A_\mathbb{C} \in \mathbb{O}(V_\mathbb{C)}$

\subsection*{Собственные значения и собственные вектора
 унитарного оператора}
Пусть $A_\mathbb{C} \in \mathbb{O}(V_\mathbb{C)}$, тогда: \newline
$(A_\mathbb{C}x,A_\mathbb{C}x) = (\lambda x,\lambda x)_\mathbb{C}= \lambda \overline{\lambda}(x,x)_\mathbb{C}= |\lambda|^2(x,x)_\mathbb{C}= (x,x)_\mathbb{C} => |\lambda$|^2 = 1 => |\lambda$| = 1\newline

Где $\lambda$ это собственное значение, $\lambda \in \mathbb{C}$, а x - собственный вектор\newline
Собственные значения унитарного оператора лежат на единичной окружности:\newline
$\lambda = e^{i\lambda} = \cos(\varphi) + i\sin(\varphi)$

\textbf{Лемма}\newline
 Собственные векторы унитарного оператора, соответствующие
 разным собственным значениям, ортогональны.\newline
 Следствие из леммы о собственных векторах ортогонального оператора

\end{document}
